\documentclass{article}
\usepackage{amsmath, amssymb, amsthm}
\usepackage{inputenc}
\usepackage{geometry}
\usepackage{graphicx}
\usepackage{setspace}
\geometry{legalpaper, portrait, margin=1in}
\title{Final Project Rough Draft}
\author{Dylan VanAllen}
\date{Wednesday \\ 3/17/2021}
\doublespacing
\begin{document}
\maketitle
\par
\indent
A spectophotometer can measure the amount of photons present, or the intensity of a beam of light. By measuring the change 
in intensity, the absorption of the light can be determined. When a photon collides with a substance, a photon with energy \(E=hf\) 
that matches the energy required for an electron in that substance to transition from one energy level to the next, it gets absorbed.
The corresponding wavelength that gets absorbed is that of the compliment of the color that our eyes see. Thus, a substance containing blue dye 
would absorb the most orange light \((\lambda_{max}\approx 650 \ nm)\). With the ability to measure absorption, we may want to use the Beer-Lambert law to 
make a calibration curve with the concentration of a substance. It follows from the equation \(A = \varepsilon \cdot \ell \cdot c\) that \(A \propto c\) with constant 
absorptivity \(\varepsilon\) and constant photon path length \(\ell\). By constructing an experiment where these are constant, we can measure the absorption of a specific 
wavelength and determine the concentration of a specific substance in a solution. If we can determine the path length \(\ell\), then 
we can experimentally determine the value of the absorptivity coefficient \(\varepsilon\) from the slope of the curve and from the path length. However, it 
is an intrinsic property of the substance and can be looked up. By rearranging the equation for Beer's law and solving for \(c\), the concentration of a given substance 
is \(c=\frac{A}{\varepsilon\ell}\). For blue dye, the absorptivity coefficient \(\varepsilon \approx 130,000\), and the concentration required for an absorbance of \(1.0 \ AU\) 
and a path length of \(1.46 \ cm\) is \(c = \frac{1}{(130000)(1.46\times 10^{-2})}=5.27\times 10^{-4} \ M\). One can use \(c_1 V_1 = c_2 V_2\) to determine the volume of 
\(2.0 \ M\) dye solution to make \(100 \ mL\) of \(5.27\times 10^{-4} \ M\) solution. \(V_1 = \frac{c_2 V_2}{c_1} = \frac{(5.27\times 10^{-4})(100)}{2.0} = 0.026 \ mL\). 
Using the Beer-Lambert law it follows that if the absorbance is divided by \(4\), the concentration would also be divided by \(4\). So the volume of solution necessary to 
cause the absorption to drop from \(1.0 \ AU\) to \(0.25 \ AU\) is found by using \(c_1 V_1 = c_2 V_2 = \frac{c_1}{4}V_2\) and rearranging, to get \(V_2 = 4V_1 = 400 \ mL\). 
\par
In the experimental data, absorbance is plotted against concentrations of blue dye. A calibration curve is made of the form \(y=mx+b\), with \(m = 190700\) and \(b=0.0007\), and now
the expected concentration of blue dye in a gatorade with known absorbance can be found. The concentration of blue dye in \(0.451 \ AU\) Gatorade is \(x = \frac{y-b}{m} = \frac{0.451 -0.0007}{190700} = 2.36 \times 10^{-6}\) M.     
\end{document}